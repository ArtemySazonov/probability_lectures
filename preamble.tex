\documentclass[a4paper,reqno]{article}

\usepackage[T1,T2A]{fontenc}
\usepackage[utf8]{inputenc}
\usepackage[english,russian]{babel}
\usepackage{amsmath,amsthm,amssymb}
\usepackage{mathtext}
\usepackage{amsthm}
\usepackage{graphicx}
\usepackage{cmap}
\usepackage{bbm}
\usepackage{tikz}

% Types

\newtheorem{theorem}{Теорема}
\newtheorem{lemma}{Лемма}
\newtheorem{prop}{Утверждение}
\newtheorem{col}{Следствие}

\theoremstyle{definition}
\newtheorem{definition}{Определение}

\theoremstyle{definition}
\newtheorem{example}{Пример}

\theoremstyle{definition}
\newtheorem{nb}{Замечание}

\newcommand{\hm}[1]{#1\nobreak\discretionary{}{\hbox{\ensuremath{#1}}}{}}
\newcommand{\nn}{\nonumber}
\DeclareMathOperator{\sgn}{sign}
\DeclareMathOperator*{\E}{\mathbb{E}}
\DeclareMathOperator*{\var}{var}   
\DeclareMathOperator*{\Var}{Var}     
\DeclareMathOperator*{\cov}{cov}

\newcommand{\1}{\mathbbm{1}} %Indicator: \1 (X<5)
\newcommand{\Co}{\mathbb{C}}
\newcommand{\R}{\mathbb{R}}
\newcommand{\N}{\mathbb{N}}
\renewcommand{\P}{P}
\newcommand{\eps}{\varepsilon}
\renewcommand{\phi}{\varphi}

\numberwithin{equation}{section}
\numberwithin{theorem}{section}
\numberwithin{lemma}{section}
\numberwithin{prop}{section}
\numberwithin{col}{section}
\numberwithin{definition}{section}
\numberwithin{example}{section}
\numberwithin{nb}{section}

\usepackage{hyperref}
\hypersetup{
	colorlinks=true,
	linkcolor=blue,
	filecolor=magenta,      
	urlcolor=cyan,
}

\date{Весенний семестр 2021}
\title{Курс <<Теория вероятностей>>}
\author{Александр Вадимович Булинский}
